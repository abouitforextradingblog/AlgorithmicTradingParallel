% !TEX TS-program = pdflatex
% !TEX encoding = UTF-8 Unicode

% This is a simple template for a LaTeX document using the "article" class.
% See "book", "report", "letter" for other types of document.

\documentclass[11pt,english]{article} % use larger type; default would be 10pt

\usepackage[utf8]{inputenc} % set input encoding (not needed with XeLaTeX)

%%% Examples of Article customizations
% These packages are optional, depending whether you want the features they provide.
% See the LaTeX Companion or other references for full information.

%%% PAGE DIMENSIONS
\usepackage{geometry} % to change the page dimensions
\geometry{a4paper} % or letterpaper (US) or a5paper or....
% \geometry{margin=2in} % for example, change the margins to 2 inches all round
% \geometry{landscape} % set up the page for landscape
%   read geometry.pdf for detailed page layout information

\usepackage{graphicx} % support the \includegraphics command and options

% \usepackage[parfill]{parskip} % Activate to begin paragraphs with an empty line rather than an indent

%%% PACKAGES
\usepackage{booktabs} % for much better looking tables
\usepackage{array} % for better arrays (eg matrices) in maths
\usepackage{paralist} % very flexible & customisable lists (eg. enumerate/itemize, etc.)
\usepackage{verbatim} % adds environment for commenting out blocks of text & for better verbatim
\usepackage{subfig} % make it possible to include more than one captioned figure/table in a single float
% These packages are all incorporated in the memoir class to one degree or another...

%%% HEADERS & FOOTERS
\usepackage{fancyhdr} % This should be set AFTER setting up the page geometry
\pagestyle{fancy} % options: empty , plain , fancy
\renewcommand{\headrulewidth}{0pt} % customise the layout...
\lhead{}\chead{}\rhead{}
\lfoot{}\cfoot{\thepage}\rfoot{}

%%% SECTION TITLE APPEARANCE
\usepackage{sectsty}
\allsectionsfont{\sffamily\mdseries\upshape} % (See the fntguide.pdf for font help)
% (This matches ConTeXt defaults)

%%% ToC (table of contents) APPEARANCE
\usepackage[nottoc,notlof,notlot]{tocbibind} % Put the bibliography in the ToC
\usepackage[titles,subfigure]{tocloft} % Alter the style of the Table of Contents
\renewcommand{\cftsecfont}{\rmfamily\mdseries\upshape}
\renewcommand{\cftsecpagefont}{\rmfamily\mdseries\upshape} % No bold!

%%% END Article customizations

%%% The "real" document content comes below...
\usepackage{babel}

\title{Parallel Programming in Algorithmic Trading - A Proposal}
\author{
  Atheendra Tarun\\
  \texttt{ap778@cornell.edu}
  \and
  Harsh Pandey\\
  \texttt{hp349@cornell.edu}
  \and
  Santoshkalyan R\\
  \texttt{scr96@cornell.edu}
  \and
  Shashank Adimulam\\
  \texttt{sa793@cornell.edu}
  \and
  Prasannjit Kumar\\
  \texttt{pk435@cornell.edu}
}
%\date{} % Activate to display a given date or no date (if empty),
         % otherwise the current date is printed 

\begin{document}
\maketitle

\section{Problem Statement}

Our project aims to study and understand the effects of parallel computing in real world applications. An ideal example of a computationally-intensive problem that would benefit substantially by utilizing multiple processors would be algorithmic trading. The problem, here, is to be able to process a huge number of stock prices within milliseconds, so that decisions related to buying or selling the stock can be taken in the least amount of time possible, to maximize profits. There are a multitude of algorithms for trading, but for demonstration, we will pick the Moving Average Convergence/Divergence (abbreviated as $MACD$) algorithm, which belongs to the overarching class of Moving Average trading techniques.

The MACD signal is generated by calculating the average of the previous $n$ data points, which represents the "momentum" of the stock price. When the stock price crosses the MACD signal in either upward or downward direction, we can surmise that there is a disparity between the current price of the stock and that of the potential future price. When the stock signal crosses MACD upwards, in other words when the stock signal cuts the MACD and obtains a value higher than that of the MACD, it generates a sell signal. If the reverse happens, we generate a buy signal. While there are other variations of this algorithm, this method is sufficient to demonstrate the advantages of parallel execution.

\subsection{Tasks involved}

\begin{itemize}

	\item Create a serial implementation of MACD for comparison.

	\item Create several variations of parallel implementations for identifying the best technique. This includes:
	\begin{itemize}
		\item Allocate a complete stock to each processor. This eliminates advantages gained through caching. However, there is almost no necessity for communication between the processors.
		\item Distribute work load based on time period. Each processor works on calculating the average of only a certain time frame of a single stock, rather than on a stock-by-stock basis. We predict that this performs slightly better in terms of memory coherency. 
		\item Explore multiple time periods for the MACD signal by calculating moving averages for different number of data points.
	\end{itemize}

	\item Compare ideal speedup with actual speedup and perform tuning and optimizations.

\end{itemize}

\end{document}
